\documentclass[10pt,oneside]{article}
\input{style/coursHeadings}

\usepackage{style/schemabloc}
%Si le boolen xp est vrai : compilation pour xabi
%Sinon compilation Damien
\newboolean{xp}
\setboolean{xp}{true}

\newboolean{prof}
\setboolean{prof}{true}

\def\xxtitre{\ifthenelse{\boolean{xp}}{
CI 2 -- SLCI : Étude du comportement des Systèmes Linéaires Continus Invariants}{
}}

\def\xxsoustitre{\ifthenelse{\boolean{xp}}{
Chapitre 7 -- Tracer des réponses harmoniques -- Diagramme de Bode}{
}}


\def\xxauteur{\ifthenelse{\boolean{xp}}{
\noindent 2013 -- 2014 \\
Xavier \textsc{Pessoles}}{
}}


\def\xxpied{\ifthenelse{\boolean{xp}}{
CI 2 : SLCI -- Cours \\
Ch 7 : Réponses harmoniques -- \ifthenelse{\boolean{prof}}{P}{E}%
}{
}}

\usepackage[%
    pdftitle={SLCI - Réponses harmoniques},
    pdfauthor={Xavier Pessoles},
    colorlinks=true,
    linkcolor=blue,
    citecolor=magenta]{hyperref}



\usepackage{pifont}
\sloppy
\hyphenpenalty 10000


\begin{document}


\input{style/entete1}

\begin{center}
 \huge\textsc{\xxtitre}
\end{center}

\begin{center}
 \LARGE\textsc{\xxsoustitre}
\end{center}

\vspace{.5cm}

\begin{savoir}
\textbf{Savoirs :}
\begin{itemize}
\item Mod.
\end{itemize}
\end{savoir}




\setlength{\parskip}{0ex plus 0.2ex minus 0ex}
 \renewcommand{\contentsname}{}
 \renewcommand{\baselinestretch}{1}

\tableofcontents

 \renewcommand{\baselinestretch}{1.2}
\setlength{\parskip}{2ex plus 0.5ex minus 0.2ex}

% \vspace{1cm}
\textit{Ce document évolue. Merci de signaler toutes erreurs ou coquilles.}



\section{Présentation}
\subsection{Caractérisation d'un signal sinusoïdal}

\subsection{Réponse temporelle à une entrée sinusoïdale}
\section{Diagramme de Bode}

\subsection{Définition}

\subsection{Représentation graphique}

\subsection{Représentation d'un système asservi}

\section{Réponse harmonique d'un gain}

\section{Réponse harmonique d'un intégrateur}

\subsection{Réponse harmonique système proportionnel intégral}

\section{Réponse harmonique d'un système du premier ordre}

\section{Réponse harmonique d'un système du second ordre}

\subsection{Cas où $\xi>1$}
\subsection{Cas où $\xi=1$}
\subsection{Cas où $\xi<1$}
\section{Réponse harmonique d'un système asservi}

\begin{thebibliography}{2}
   %\bibitem[1]{cite1} DMU 60 eVo linear, \textit{DMG -- Deckel Maho -- Gildemeiseter}, \url{http://fr.dmg.com}.
   %\bibitem[2]{cite2} Programmation des machines-outils à commande numérique (MOCN), \textit{Étienne Lefur et Christophe Sohier}, École Normale Supérieure de Cachan, \url{http://etienne.lefur.free.fr/}.
   %\bibitem[3]{cite3} SLCI : Systèmes asservis en boucle fermée : stabilité et précision, \textit{Joël Boiron}, PTSI -- Lycée Gustave Eiffel de Bordeaux.

\end{thebibliography}

\end{document}
