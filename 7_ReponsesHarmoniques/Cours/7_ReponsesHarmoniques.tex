\documentclass[10pt,oneside]{article}
\input{style/coursHeadings}

\usepackage{style/schemabloc}
%Si le boolen xp est vrai : compilation pour xabi
%Sinon compilation Damien
\newboolean{xp}
\setboolean{xp}{true}

\newboolean{prof}
\setboolean{prof}{true}

\def\xxtitre{\ifthenelse{\boolean{xp}}{
CI 2 -- SLCI : Étude du comportement des Systèmes Linéaires Continus Invariants}{
}}

\def\xxsoustitre{\ifthenelse{\boolean{xp}}{
Chapitre 7 -- Réponses harmoniques -- Diagrammes de Bode}{
}}


\def\xxauteur{\ifthenelse{\boolean{xp}}{
\noindent 2013 -- 2014 \\
Xavier \textsc{Pessoles}}{
}}


\def\xxpied{\ifthenelse{\boolean{xp}}{
CI 2 : SLCI -- Cours \\
Ch 7 : Réponses harmoniques -- \ifthenelse{\boolean{prof}}{P}{E}%
}{
}}

\usepackage[%
    pdftitle={SLCI - Réponses harmoniques},
    pdfauthor={Xavier Pessoles},
    colorlinks=true,
    linkcolor=blue,
    citecolor=magenta]{hyperref}



\usepackage{pifont}
\sloppy
\hyphenpenalty 10000


\begin{document}


\input{style/entete1}

\begin{center}
 \huge\textsc{\xxtitre}
\end{center}

\begin{center}
 \LARGE\textsc{\xxsoustitre}
\end{center}

\vspace{.5cm}

\begin{savoir}
\textbf{Savoirs :}
\begin{itemize}
\item Mod.
\end{itemize}
\end{savoir}




\setlength{\parskip}{0ex plus 0.2ex minus 0ex}
 \renewcommand{\contentsname}{}
 \renewcommand{\baselinestretch}{1}

\tableofcontents

 \renewcommand{\baselinestretch}{1.2}
\setlength{\parskip}{2ex plus 0.5ex minus 0.2ex}

% \vspace{1cm}
\textit{Ce document évolue. Merci de signaler toutes erreurs ou coquilles.}



\section{Présentation}
\subsection{Caractérisation d'un signal sinusoïdal}

On peut définir un signal sinusoïdal sous la forme suivante :
$$
f(t)=A \sin(\omega \cdot t + \varphi)
$$
et on note :
\begin{itemize}
\item $A$ : l'amplitude de la sinusoïde;
\item $\omega$ : la pulsation en $rad/s$;
\item $\varphi$ : la phase à l'origine en $rad$.
\end{itemize}
On a par ailleurs :
\begin{itemize}
\item $T=\dfrac{2\pi}{\omega}$ : la période de la sinusoïde en $s$;
\item $f=\dfrac{1}{T}$ : fréquence de la sinusoïde en $Hz$.
\end{itemize}


\subsection{Réponse temporelle à une entrée sinusoïdale}
\begin{defi}
\textbf{Réponse harmonique}

On appelle réponse harmonique la sortie d'un système lorsqu'il est soumis à une entrée sinusoïdale. Elle permet de caractériser le comportement dynamique du système.
\end{defi}

\section{Diagrammes de Bode}

\subsection{Calcul complexe}
Lorsque le système est soumis à une entrée sinusoïdale, la variable de la Laplace $p$ est substituée par $j\omega$. $H(j\omega)$ est appelée réponse en fréquence ou réponse harmonique du système.

\begin{resultat}
On rappelle que si $H(j\omega) = \dfrac{x_1+jy_1}{x_2+jy_2}$, alors : 
\begin{itemize}
\item on appelle $A(\omega)=|H(j\omega)|$ le module de $H$ (ou le gain de $H$) et on a : 
$$
A(\omega)= \dfrac{\sqrt{x_1^2+y_1^2}}{\sqrt{x_2^2+y_2^2}}
$$
\item on appelle $\varphi(\omega)=Arg(H(j\omega))$ l'argument de $H$ (ou la phase de $H$) et on a : 
$$
\varphi(\omega)= \arctan \dfrac{y_1}{x_1}-\arctan \dfrac{y_2}{x_2}
$$
\end{itemize}
\end{resultat}

\begin{rem}
On appelle $Adb$ le gain en décibel et on a :
$$
Adb(\omega)=20 \log A(\omega)
$$
\end{rem}

\begin{exemple}
Soit $H(p)$ une fonction de transfert d'ordre 1 :
$$
H(j\omega)= \dfrac{K}{1+\tau j\omega }
$$

On a alors : 
$$
Adb(\omega) = 20 \log \left(\dfrac{\sqrt{K^2}}{\sqrt{1^2+(\tau\omega)^2}}\right) = 20 \log K - 10 \log \left(1+\tau^2\omega^2 \right)
$$
$$
\varphi(\omega)= \arctan \dfrac{0}{K}-\arctan \dfrac{\tau\omega}{1} = - \arctan \tau\omega
$$

\end{exemple}

\subsection{Définition}

\begin{defi}
\textbf{Diagrammes de Bode}

Les diagrammes de Bode représentent deux courbes sur deux diagrammes distincts dans des repère semi logarithmiques :
\begin{itemize}
\item la courbe de gain en décibel en fonction de la pulsation $\omega$;
\item la courbe de phase (en degrés ou radians) en fonction de la pulsation $\omega$.
\end{itemize}

\end{defi}


\subsection{Représentation d'un système asservi}

Généralement, une fonction de transfert s'écrit sous la forme d'un produit de fonction rationnelles. Ainsi, notons $H(j\omega)=F(j\omega) \cdot G(j\omega)$. 

On montre que le gain décibel de $H$ est sous la forme :
$$
Adb(\omega) = 20 \log |F(j\omega) |+20 \log |G(j\omega) |
$$

et que la phase est sous la forme :
$$
Adb(\omega) = Arg \left(F(j\omega) \right)+Arg \left(G(j\omega) \right)
$$

Ainsi pour tracer le gain et la phase d'une fonction de transfert s'exprimant sous la forme de produit de fonctions de transfert élémentaire, il suffit de tracer les fonctions de transfert élémentaire dans les diagrammes de Bode puis de les sommer.

\section{Réponse harmonique d'un gain}

\section{Réponse harmonique d'un intégrateur}

\subsection{Réponse harmonique système proportionnel intégral}

\section{Réponse harmonique d'un système du premier ordre}

\section{Réponse harmonique d'un système du second ordre}

\subsection{Cas où $\xi>1$}
\subsection{Cas où $\xi=1$}
\subsection{Cas où $\xi<1$}
\section{Réponse harmonique d'un système asservi}

\begin{thebibliography}{2}
   %\bibitem[1]{cite1} DMU 60 eVo linear, \textit{DMG -- Deckel Maho -- Gildemeiseter}, \url{http://fr.dmg.com}.
   %\bibitem[2]{cite2} Programmation des machines-outils à commande numérique (MOCN), \textit{Étienne Lefur et Christophe Sohier}, École Normale Supérieure de Cachan, \url{http://etienne.lefur.free.fr/}.
   %\bibitem[3]{cite3} SLCI : Systèmes asservis en boucle fermée : stabilité et précision, \textit{Joël Boiron}, PTSI -- Lycée Gustave Eiffel de Bordeaux.

\end{thebibliography}

\end{document}
